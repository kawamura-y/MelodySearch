\documentclass[11pt]{jreport}
\usepackage{latexsym}
\usepackage{mathrsfs}
\usepackage{amssymb}
%\usepackage{url}
%\usepackage{lscape}
\usepackage{graphics}
\usepackage{theorem}

%% for apple LaserWriter Series %%
%% 
\setlength{\topmargin}{-0.5in}
\setlength{\textwidth}{5.6in}
\setlength{\textheight}{8.8in}
\setlength{\oddsidemargin}{0.35in}
\setlength{\evensidemargin}{0in}

\usepackage{theorem}
\renewcommand{\baselinestretch}{1.25}
\setlength{\parskip}{0.25ex}
\renewcommand{\arraystretch}{0.85}
\begin{document}


\subsection*{定義}
musical note-on sequence $\mathit{\mu} = \langle e_1, \ldots, e_n \rangle$ は組 $e_i = (t_i, p_i, \nu_i) \in \mathbb{Z}^+ \times \{1, \ldots, 15\} \times \{0, \ldots, 127\}$ の列である.
$t_i$ を発音時刻,$p_i$ をプログラム,$\nu_i$ をノートナンバー(音番号)という.
デジタル楽譜または SMF ファイルは,musical note-on sequence ひとつ,またはそれらいくつかの組とみなす.

musical note-on sequence の subsequence で,
あるプログラム $p$ を持つ組すべてを持つもの $t_{\pi(1)}, \ldots, t_{\pi(m)}$ ($1 \leq \pi(1) < \cdots< \pi(m) \leq n$)を単一プログラム列という.

(要改良)
単一プログラム列から,
各発音時刻について(発音時刻が同じ組の)ノートナンバーのうち最も大きい組(1つしかないばあいはその組)を取り出した列の,
連続する(隣り合う)ノートナンバーの差 $\Delta_{p,i} = \nu_{\pi(i+1)} - \nu_{\pi(i)}$ の列を,音程列 interval sequence という.
音程列は,有限アルファベット $I = \{-127, \ldots, 0, \ldots, 127\}$ 上の文字列である.

\begin{defn}[音程マッチング, メロディーコンター?マッチング]
音程列マッチングは,与えられた長さが異なる音程列 $s, t \in I^*$ ただし $|s| \leq |t|$ に対して,$s$ が $t$ 中で最初に出現する位置,すなわち
$0 \leq i < |s|$ のとき $t[l+i] = s[i]$ となる(最小の)$l$ を見つける(またはみつからないことを確認する)問題である.

(最初の基本形)
Melodic contour とは,音程列のすべての文字を, $-127$ から $-1$ は $\mathtt{-}$に, $0$ は $\mathtt{=}$に, $1$ から $127$ は $\mathtt{+}$ に置き換えた文字列のことである.
Melodic contour matching は,与えられた melodic contour $s, t \in \{\mathtt{-}, \mathtt{=}, \mathtt{+}\}$ に対して,$s$ が $t$ 中に最初に出現する位置をみつける問題である.
\end{defn}

\end{document}
